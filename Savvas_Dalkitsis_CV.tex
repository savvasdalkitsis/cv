\documentclass[10pt, letterpaper]{article}

% Packages:
\usepackage[
    ignoreheadfoot, % set margins without considering header and footer
    top=2 cm, % seperation between body and page edge from the top
    bottom=2 cm, % seperation between body and page edge from the bottom
    left=2 cm, % seperation between body and page edge from the left
    right=2 cm, % seperation between body and page edge from the right
    footskip=1.0 cm, % seperation between body and footer
    % showframe % for debugging 
]{geometry} % for adjusting page geometry
\usepackage[explicit]{titlesec} % for customizing section titles
\usepackage{tabularx} % for making tables with fixed width columns
\usepackage{array} % tabularx requires this
\usepackage[dvipsnames]{xcolor} % for coloring text
\definecolor{primaryColor}{RGB}{0, 79, 144} % define primary color
\usepackage{enumitem} % for customizing lists
\usepackage{fontawesome5} % for using icons
\usepackage{amsmath} % for math
\usepackage[
    pdftitle={Savvas Dalkitsis's CV},
    pdfauthor={Savvas Dalkitsis},
    pdfcreator={LaTeX with RenderCV},
    colorlinks=true,
    urlcolor=primaryColor
]{hyperref} % for links, metadata and bookmarks
\usepackage[pscoord]{eso-pic} % for floating text on the page
\usepackage{calc} % for calculating lengths
\usepackage{bookmark} % for bookmarks
\usepackage{lastpage} % for getting the total number of pages
\usepackage{changepage} % for one column entries (adjustwidth environment)
\usepackage{paracol} % for two and three column entries
\usepackage{ifthen} % for conditional statements
\usepackage{needspace} % for avoiding page brake right after the section title
\usepackage{iftex} % check if engine is pdflatex, xetex or luatex

% Ensure that generate pdf is machine readable/ATS parsable:
\ifPDFTeX
    \input{glyphtounicode}
    \pdfgentounicode=1
    \usepackage[T1]{fontenc}
    \usepackage[utf8]{inputenc}
    \usepackage{lmodern}
\fi

\usepackage[default, type1]{sourcesanspro} 

% Some settings:
\AtBeginEnvironment{adjustwidth}{\partopsep0pt} % remove space before adjustwidth environment
\pagestyle{empty} % no header or footer
\setcounter{secnumdepth}{0} % no section numbering
\setlength{\parindent}{0pt} % no indentation
\setlength{\topskip}{0pt} % no top skip
\setlength{\columnsep}{0.15cm} % set column seperation
\makeatletter
\let\ps@customFooterStyle\ps@plain % Copy the plain style to customFooterStyle
\patchcmd{\ps@customFooterStyle}{\thepage}{
    \color{gray}\textit{\small Savvas Dalkitsis - Page \thepage{} of \pageref*{LastPage}}
}{}{} % replace number by desired string
\makeatother
\pagestyle{customFooterStyle}

\titleformat{\section}{
    % avoid page braking right after the section title
    \needspace{4\baselineskip}
    % make the font size of the section title large and color it with the primary color
    \Large\color{primaryColor}
}{
}{
}{
    % print bold title, give 0.15 cm space and draw a line of 0.8 pt thickness
    % from the end of the title to the end of the body
    \textbf{#1}\hspace{0.15cm}\titlerule[0.8pt]\hspace{-0.1cm}
}[] % section title formatting

\titlespacing{\section}{
    % left space:
    -1pt
}{
    % top space:
    0.3 cm
}{
    % bottom space:
    0.2 cm
} % section title spacing

% \renewcommand\labelitemi{$\vcenter{\hbox{\small$\bullet$}}$} % custom bullet points
\newenvironment{highlights}{
    \begin{itemize}[
        topsep=0.10 cm,
        parsep=0.10 cm,
        partopsep=0pt,
        itemsep=0pt,
        leftmargin=0.4 cm + 10pt
    ]
}{
    \end{itemize}
} % new environment for highlights

\newenvironment{highlightsforbulletentries}{
    \begin{itemize}[
        topsep=0.10 cm,
        parsep=0.10 cm,
        partopsep=0pt,
        itemsep=0pt,
        leftmargin=10pt
    ]
}{
    \end{itemize}
} % new environment for highlights for bullet entries


\newenvironment{onecolentry}{
    \begin{adjustwidth}{
        0.2 cm + 0.00001 cm
    }{
        0.2 cm + 0.00001 cm
    }
}{
    \end{adjustwidth}
} % new environment for one column entries

\newenvironment{twocolentry}[2][]{
    \onecolentry
    \def\secondColumn{#2}
    \setcolumnwidth{\fill, 4.5 cm}
    \begin{paracol}{2}
}{
    \switchcolumn \raggedleft \secondColumn
    \end{paracol}
    \endonecolentry
} % new environment for two column entries

\newenvironment{threecolentry}[3][]{
    \onecolentry
    \def\thirdColumn{#3}
    \setcolumnwidth{1 cm, \fill, 4.5 cm}
    \begin{paracol}{3}
    {\raggedright #2} \switchcolumn
}{
    \switchcolumn \raggedleft \thirdColumn
    \end{paracol}
    \endonecolentry
} % new environment for three column entries

\newenvironment{header}{
    \setlength{\topsep}{0pt}\par\kern\topsep\centering\color{primaryColor}\linespread{1.5}
}{
    \par\kern\topsep
} % new environment for the header

\newcommand{\placelastupdatedtext}{% \placetextbox{<horizontal pos>}{<vertical pos>}{<stuff>}
  \AddToShipoutPictureFG*{% Add <stuff> to current page foreground
    \put(
        \LenToUnit{\paperwidth-2 cm-0.2 cm+0.05cm},
        \LenToUnit{\paperheight-1.0 cm}
    ){\vtop{{\null}\makebox[0pt][c]{
        \small\color{gray}\textit{Last updated in July 2024}\hspace{\widthof{Last updated in July 2024}}
    }}}%
  }%
}%

% save the original href command in a new command:
\let\hrefWithoutArrow\href

% new command for external links:
\renewcommand{\href}[2]{\hrefWithoutArrow{#1}{\ifthenelse{\equal{#2}{}}{ }{#2 }\raisebox{.15ex}{\footnotesize \faExternalLink*}}}


\begin{document}
    \newcommand{\AND}{\unskip
        \cleaders\copy\ANDbox\hskip\wd\ANDbox
        \ignorespaces
    }
    \newsavebox\ANDbox
    \sbox\ANDbox{}

    \placelastupdatedtext
    \begin{header}
        \fontsize{30 pt}{30 pt}
        \textbf{Savvas Dalkitsis}

        \vspace{0.3 cm}

        \normalsize
        \mbox{{\footnotesize\faMapMarker*}\hspace*{0.13cm}London, UK}%
        \kern 0.25 cm%
        \AND%
        \kern 0.25 cm%
        \mbox{\hrefWithoutArrow{mailto:kurosavvas@gmail.com}{{\footnotesize\faEnvelope[regular]}\hspace*{0.13cm}kurosavvas@gmail.com}}%
        \kern 0.25 cm%
        \AND%
        \kern 0.25 cm%
        \mbox{\hrefWithoutArrow{tel:+44-7903-432695}{{\footnotesize\faPhone*}\hspace*{0.13cm}+44 7903 432695}}%
        \kern 0.25 cm%
        \AND%
        \kern 0.25 cm%
        \mbox{\hrefWithoutArrow{https://card.savvas.cloud/}{{\footnotesize\faLink}\hspace*{0.13cm}card.savvas.cloud}}%
        \kern 0.25 cm%
        \AND%
        \kern 0.25 cm%
        \mbox{\hrefWithoutArrow{https://linkedin.com/in/savvasdalkitsis}{{\footnotesize\faLinkedinIn}\hspace*{0.13cm}savvasdalkitsis}}%
        \kern 0.25 cm%
        \AND%
        \kern 0.25 cm%
        \mbox{\hrefWithoutArrow{https://github.com/savvasdalkitsis}{{\footnotesize\faGithub}\hspace*{0.13cm}savvasdalkitsis}}%
    \end{header}

    \vspace{0.3 cm - 0.3 cm}


    \section{About Me}



        
        \begin{onecolentry}
            Android geek, TDD nut, international speaker, born in Greece, British citizen living in London for the past 12 years.
        \end{onecolentry}

        \vspace{0.2 cm}

        \begin{onecolentry}
            I'm obsessed with technology, and love everything Android. I have been coding as a pastime since the age of 7 and have been doing it professionally since 2008, focusing on Android for the last 13 years.
        \end{onecolentry}

        \vspace{0.2 cm}

        \begin{onecolentry}
            I have a B.Math from Aristotle University in Thessaloniki, Greece
        \end{onecolentry}


    
    \section{Experience}



        
        \begin{twocolentry}{
            London, UK

        Aug 2017 – Aug 2024

        7 years 1 month
        }
            \textbf{ASOS}, Principal Software Engineer
            \begin{highlights}
                \item The Apps team (iOS and Android) grew at a very fast pace, from 10 engineers to more than 30 in the span of 2 years, and someone was needed to take ownership of the team as a whole. In my new role as Principal Engineer , I moved away from the daily delivery of one agile team into having an overview of all 6 teams on both platforms. I was tasked with providing support and guidance when it came to code quality while also focusing on the bigger picture of how the teams deliver, making the developer experience as removed from distractions as possible. Notable work includes
                \item Helping bring the iOS team to the same level of automation as the Android team
                \item Moving all new development on the Android client to Kotlin, greatly reducing maintenance overhead and bugs. A talk about our transition was presented at \href{https://noti.st/savvas/3RYKMi}{Mobiconf, Kraków}
                \item Helping our Principal Architect and Platform Leads plan the upcoming roadmaps by being involved during the research phase and helping estimate effort and feasibility of proposed features
                \item Deeply involved in helping shape our recruitment process, screening candidates before the in-depth technical interview
                \item Getting involved in the wider engineering community inside ASOS, something the Apps team had been very isolated from, and started adopting some of the tools \& standards already in place for the other teams
                \item Building a new configuration web service for the apps, allowing us to easily and quickly configure our offering per region and market. The service is deployed using Docker and Microsoft's AKS
                \item Internally evangelizing the Apps team's testing methodology, with a few teams having already adopted our approach
                \item Being heavily involved in the reboot of the \href{https://medium.com/asos-techblog}{ASOS technology blog}
            \end{highlights}
        \end{twocolentry}


        \vspace{0.2 cm}

        \begin{twocolentry}{
            London

        July 2016 – Aug 2017

        1 year 1 month
        }
            \textbf{ASOS}, Lead Software Engineer
            \begin{highlights}
                \item Initially brought in to lead one of the 2 Android teams, I was responsible for 4 engineers, helping maintain coding standards and making sure the team could deliver in a timely manner while not sacrificing quality. During my time as a team lead
                \item I introduced a test first mentality, focusing on BDD and automated testing, which the project lacked completely
                \item Mentored junior and senior team members, especially with regards to test driven development
                \item Having achieved almost 100\% acceptance testing automation, we managed to reduce our release cadence. from 1 month with a half day regression testing phase, down to a weekly release with a 10 minute smoke test. We have effectively achieved CD, only choosing a weekly release schedule so as to not swamp our users with updates
            \end{highlights}
        \end{twocolentry}



    
    \section{Quick Guide}

    \begin{onecolentry}
        \begin{highlightsforbulletentries}


        \item Each section title is arbitrary and each section contains a list of entries.

        \item There are 7 unique entry types: \textit{BulletEntry}, \textit{TextEntry}, \textit{EducationEntry}, \textit{ExperienceEntry}, \textit{NormalEntry}, \textit{PublicationEntry}, and \textit{OneLineEntry}.

        \item Select a section title, pick an entry type, and start writing your section!

        \item \href{https://docs.rendercv.com/user_guide/}{Here}, you can find a comprehensive user guide for RenderCV.


        \end{highlightsforbulletentries}
    \end{onecolentry}

    \section{Education}



        
        \begin{threecolentry}{\textbf{BS}}{
            Sept 2000 – May 2005
        }
            \textbf{University of Pennsylvania}, Computer Science
            \begin{highlights}
                \item GPA: 3.9/4.0 (\href{https://example.com}{Transcript})
                \item \textbf{Coursework:} Computer Architecture, Comparison of Learning Algorithms, Computational Theory
            \end{highlights}
        \end{threecolentry}


    
    \section{Publications}



        
        \begin{samepage}
            \begin{twocolentry}{
                Jan 2004
            }
                \textbf{3D Finite Element Analysis of No-Insulation Coils}

                \vspace{0.10 cm}

                \mbox{Frodo Baggins}, \mbox{\textbf{\textit{John Doe}}}, \mbox{Samwise Gamgee}
                \vspace{0.10 cm}

        \href{https://doi.org/10.1109/TASC.2023.3340648}{10.1109/TASC.2023.3340648}
            \end{twocolentry}
        \end{samepage}


    
    \section{Projects}



        
        \begin{twocolentry}{
            \href{https://github.com/sinaatalay/rendercv}{github.com/name/repo}
        }
            \textbf{Multi-User Drawing Tool}
            \begin{highlights}
                \item Developed an electronic classroom where multiple users can simultaneously view and draw on a "chalkboard" with each person's edits synchronized
                \item Tools Used: C++, MFC
            \end{highlights}
        \end{twocolentry}


        \vspace{0.2 cm}

        \begin{twocolentry}{
            \href{https://github.com/sinaatalay/rendercv}{github.com/name/repo}
        }
            \textbf{Synchronized Desktop Calendar}
            \begin{highlights}
                \item Developed a desktop calendar with globally shared and synchronized calendars, allowing users to schedule meetings with other users
                \item Tools Used: C\#, .NET, SQL, XML
            \end{highlights}
        \end{twocolentry}


        \vspace{0.2 cm}

        \begin{twocolentry}{
            2002
        }
            \textbf{Custom Operating System}
            \begin{highlights}
                \item Built a UNIX-style OS with a scheduler, file system, text editor, and calculator
                \item Tools Used: C
            \end{highlights}
        \end{twocolentry}



    
    \section{Technologies}



        
        \begin{onecolentry}
            \textbf{Languages:} C++, C, Java, Objective-C, C\#, SQL, JavaScript
        \end{onecolentry}

        \vspace{0.2 cm}

        \begin{onecolentry}
            \textbf{Technologies:} .NET, Microsoft SQL Server, XCode, Interface Builder
        \end{onecolentry}


    

\end{document}